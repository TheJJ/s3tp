\documentclass[a4paper]{article}

\usepackage[utf8]{inputenc}

% multiline comments with \nope lol text //
% or \comment{lol text}
\long\def\nope #1 //{}

\newcommand{\comment}[1]{}

\usepackage[pdftex,%
            colorlinks=true,linkcolor=blue,citecolor=red,%
            anchorcolor=red,urlcolor=blue,bookmarks=true,%
            bookmarksopen=true,bookmarksopenlevel=0,plainpages=false,%
            bookmarksnumbered=true,hyperindex=false,pdfstartview=,%
            pdfauthor={Jonas Jelten},%
            pdftitle={S3TP},%
            pdfsubject={simple stupid satellite transmission protocol},%
            pdfkeywords={},%
            pdfcreator={pdfLaTeX on JJ-Gentoo-GNU/Linux}%
]{hyperref}

\usepackage{palatino}
\usepackage[ngerman,english]{babel}
\selectlanguage{ngerman}
\usepackage{tabularx}
\usepackage{graphicx}


\begin{document}

\subsubsection{S3TP - Simple-Stupid Satellite Transport Protocol}

\texttt{S3TP} is the transport protocol for data transmission between the
on-board SoC and ground station. Basically it is designed like TCP, but adapted
for space usage.

The system is designed to work in userspace only, assuming the Linux kernel can
do all the SPI communication to the \texttt{COM} transmission board. If kernel
components need to be tweaked because of problems with SPI components, our
system will still be functional and, because being userland applications only,
compatible for future Linux releases.

\begin{figure}
    \centering
    \includegraphics[width=\textwidth]{s3tp-overview.pdf}
    \caption{S3TP concept overview}
    \label{fig:s3tpoverview}
\end{figure}

An overview of the system can be seen in figure \ref{fig:s3tpoverview}. The
central component is the \texttt{s3tpd}, the communication daemon. It is written
in \texttt{Python 3} and uses the \texttt{asyncio} module for asynchronous
interaction with the Linux kernel.

\texttt{s3tpd} runs once for each radio device. All connections over the air are
managed by it. Each connection provides reliable in-order transmission of data
packets. They are transported to the peer-\texttt{s3tpd} and passed to the
receiving tool. Thus, \texttt{s3tp} provides a transparent layer for
connection-oriented which can be replaced easily.

\texttt{s3tpc} is a \texttt{C++} library with a \texttt{C} API. The interface to
the tool is very minimal, a tool can \texttt{listen} on ports and
\texttt{connect} to listened ports on the peer. \texttt{read} and \texttt{write}
can then be done on the connection. A blocking and nonblocking mode is available
when using the library so the application remains flexible.

TODO: protocol overview, more details?

\end{document}

%%% Local Variables:
%%% mode: latex
%%% TeX-master: t
%%% End:
